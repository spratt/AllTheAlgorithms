\chapter{Graphs (revisited)}

\section{Randomized Minimum Edge-Cut}

Given an undirected graph $G = (V,E)$, a cut in $E$ is a subset $E'$
of edges $E$ such tat if we remove $E'$ from $E$, we obtain a
disconnected graph.  We are interested in the minimum cut.

Given a graph $G = (V,E)$, we can perform \emph{edge contraction} by
taking an edge $(u,v)$, remove the edge, replace both $u$ and $v$ with
a single vertex $w$.  Any edges incident to $u$ or $v$ are now
incident to $w$ instead.

Note that any cut on a graph with a contracted edge is also a cut on
the graph with the edge uncontracted.  It follows from this that we
can contract random edges until we are left with two vertices, the
edges incident to which will form a cut on the original graph.

Note also that this may not return the min cut, but we may iterate
several times to find the min cut with high probability.

Historical note: this algorithm was developed by David Karger.
