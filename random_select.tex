\chapter{Randomized Selection}

The QuickSelect algorithm is a randomized version of the selection
algorithm covered in Chapter \ref{chapter:selection}.  Given an array
of $n$ elements in arbitrary order, QuickSelect finds the $i$th
element (where $i \le n$) in linear time.

If there is only one element, return it.  Otherwise, randomly choose
an element $x$ in the array and use this to partition the elements
into those larger than the partition and those smaller.  If the final
index of $x$ is greater than $i$, recurse on the elements smaller than
$x$.  If the final index of $x$ is less than $i$, subtract the index
of $x$ from $i$, and recurse on the elements larger than $x$.

Intuitively, we expect to eliminate about half of the array at each
level of recursion.

Say that if $x$ falls within the first or last quarters of the sorted
array, we don't remove any elements.  Otherwise, we remove half of the
elements in the array.

\begin{align*}
  %
  T(n)
  &= \frac{T(n)}{2} + \frac{T \left( \frac{n}{2} \right)}{2} + \BigOh{n} \\
  2 T(n) &= T(n) + T \left( \frac{n}{2} \right) + 2 \BigOh{n} \\
  2 T(n) - T(n) &= T \left( \frac{n}{2} \right) + \BigOh{n} \\
  T(n) &= T \left( \frac{n}{2} \right) + \BigOh{n} \\
  %
\end{align*}

Which is $\BigOh{n}$ by the Master theorem.
